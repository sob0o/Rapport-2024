\mychapter{0}{Résumé}

La maintenance prédictive est cruciale aujourd'hui pour anticiper les pannes
avant qu'elles ne surviennent, permettant ainsi de réduire les coûts et le
temps de maintenance. Les moteurs électriques, largement utilisés dans
l'industrie, les transports et les usines, sont sujets à de nombreux défauts
électriques et mécaniques, rendant leur maintenance coûteuse mais
indispensable.

\medskip

Ce mémoire explore l'application du machine learning pour la maintenance
prédictive des moteurs électriques. La qualité et la diversité des données sont  des éléments clés
du machine learning. Pour enrichir notre jeu de données sur différents moteurs
présents sur le marché, nous utilisons l'intelligence artificielle générative
pour créer des données similaires aux données réelles, sous forme de séries
temporelles. Des techniques avancées telles que les auto-encodeurs
variationnels (VAE), les réseaux adverses génératifs (GAN) et les
modèles de diffusion sont employées pour générer ces données.

\medskip
Le travail réalisé a permis de produire des séries temporelles très proches des données réelles,
lesquelles sont utilisées par un modèle de classification pour prédire de manière fiable si un moteur va tomber en panne ou non.
Les résultats obtenus démontrent que notre solution est capable de générer des données de haute qualité et de prédire efficacement
les pannes des moteurs électriques, offrant ainsi une approche prometteuse pour la maintenance prédictive dans divers secteurs industriels.

\vspace{1cm}

\noindent\rule[2pt]{\textwidth}{0.5pt}

{\textbf{Mots clés :}}
Apprentissage profond, Apprentissage automatique, Réseaux neuronaux profonds, Maintenance prédictive,
Séries temporelles, Réseaux adverses génératifs (GAN), Auto-encodeurs variationnels (VAE), les
modèles de diffusion.
\\
\noindent\rule[2pt]{\textwidth}{0.5pt}

\clearpage

\mychapter{0}{Abstract}

Predictive maintenance is crucial today for anticipating failures before they
occur, thereby reducing costs and maintenance time. Electric motors, widely
used in industry, transportation, and factories, are subject to numerous
electrical and mechanical faults, making their maintenance costly but
indispensable.

\medskip

This thesis explores the application of machine learning for the predictive
maintenance of electric motors. Data quality  and diversity  are a key elements of machine
learning. To enrich our dataset on various motors available on the market, we
use generative artificial intelligence to create data similar to real data, in
the form of time series. Advanced techniques such as variational autoencoders
(VAE), generative adversarial networks (GAN), and diffusion models 
are employed to generate these data.

\medskip

The work carried out has produced time series very close to real data, which
are used by a classification model to reliably predict whether a motor will
fail or not. The results obtained demonstrate that our solution is capable of
generating high-quality data and effectively predicting electric motor
failures, thus offering a promising approach for predictive maintenance in
various industrial sectors.

\vspace{1cm}

\noindent\rule[2pt]{\textwidth}{0.5pt}

{\textbf{Keywords :}}
Deep learning, Machine learning, Deep neural networks, Predictive maintenance,
Time series, Generative adversarial networks (GAN), Variational autoencoders (VAE), diffusion models.
\\

\noindent\rule[2pt]{\textwidth}{0.5pt}

