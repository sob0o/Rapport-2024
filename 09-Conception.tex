\part{Contribution}





\chapter{Organisme d’accueil}



\section{Introduction}
Dans ce chapitre, nous présenterons l’organisme qui nous a accueillis pour la réalisation de notre projet de fin d’études. 
Nous commencerons par une brève introduction de l’entreprise, en mettant en évidence son domaine d’activité et son positionnement 
sur le marché. Ensuite, nous détaillerons les différents services proposés par l’organisme, ainsi que son organigramme organisationnel.
Cette présentation nous permettra de mieux comprendre l’environnement dans lequel notre projet s’est déroulé et les contraintes
spécifiques auxquelles nous avons été confrontés.



\section{Présentation de l'organisme d'accueil}
STIE, ou Schneider Toshiba Inverter Europe SAS, est un département relié à Schneider Electric. Ce département offre une
gamme complète d'équipements électroniques industriels. Il propose des inverseurs polyvalents, des variateurs de vitesse 
industriels, des composants de contrôle, et des produits de distribution électrique. STIE dispose également de laboratoires 
équipés de divers matériels, tels que des moteurs électriques et leurs dérivés, pour répondre aux besoins de ses clients à travers
le monde. STIE est principalement basé à Pacy-sur-Eure et à Rueil-Malmaison.

\textbf{Schneider Electric} est un leader mondial de la technologie industrielle avec une expertise de référence dans l'électrification, l'automatisation, et la digitalisation des industries intelligentes, des infrastructures résilientes, des centres de données durables, des bâtiments intelligents, et des maisons intuitives. L'entreprise mène la transformation numérique de la gestion de l'énergie et des automatismes dans les secteurs résidentiel, des bâtiments, des centres de données, des infrastructures et des industries.
Présente dans plus de 115 pays, Schneider Electric est le leader incontesté de la gestion électrique, couvrant la moyenne tension, la basse tension, l'énergie sécurisée, et les systèmes d'automatismes. La société fournit des solutions d'efficacité intégrées qui associent gestion de l'énergie, automatismes, et logiciels. L'écosystème qu'elle a construit lui permet de collaborer sur sa plateforme ouverte avec une large communauté de partenaires, d'intégrateurs, et de développeurs pour offrir à ses clients à la fois contrôle et efficacité opérationnelle en temps réel.
La répartition géographique de son chiffre d'affaires est la suivante :

\begin{itemize}
  \item France : 5,8\%
  \item Europe de l'Ouest : 18,5\%
  \item États-Unis : 27,9\%
  \item Amérique du Nord : 4,2\%
  \item Chine : 15,1\%
  \item Asie-Pacifique : 15,2\%
  \item Autres : 13,3\%
\end{itemize}


\section{Services}

Schneider Electric propose une gamme étendue de produits et services innovants, notamment :
\begin{itemize}
    \item \textbf{Produits d'automatisation et de contrôle} : Solutions complètes pour l'automatisation des processus et la gestion des systèmes de contrôle industriel.
    \item \textbf{Produits et systèmes basse tension} : Équipements pour la distribution électrique et la protection des installations basse tension.
    \item \textbf{Énergie solaire et stockage} : Solutions pour la production d'énergie solaire et le stockage d'énergie.
    \item \textbf{Distribution moyenne tension et automatisation du réseau} : Systèmes pour la distribution d'énergie moyenne tension et l'automatisation des réseaux électriques.
    \item \textbf{Énergie critique, refroidissement et racks} : Solutions pour la gestion de l'énergie critique, le refroidissement des infrastructures et les racks de serveur.
\end{itemize}

\section{Organigramme de l'entreprise}
\section{Conclusion}
Dans ce chapitre, nous avons présenté l’organisme d’accueil qui nous a permis de réaliser notre projet de fin d’études. Nous avons fourni une description détaillée de l’entreprise, de ses activités principales et de son positionnement sur le marché. Nous avons également mis en évidence les différents services proposés par l’organisme, ainsi que son organigramme organisationnel.

Cette présentation nous a permis de mieux comprendre l’environnement dans lequel notre projet s’est déroulé et les contraintes spécifiques auxquelles nous avons été confrontés. Les 











\chapter{Conception}

%\newpage

\section{Introduction}
Le domaine de l'apprentissage profond a connu une grande croissance de la profondeur des architectures de réseaux de neurones. Cependant, cette croissante pose des défis importants en termes d'exigences de calcul et mémoire et de consommation d'énergie. Ainsi, la nécessité de concevoir des techniques efficaces de compression et d'optimisation des modèles est devenue primordiale. Dans ce chapitre, nous présenterons une méthode automatique pour l'elagage des réseaux neuronaux très profonds qui exploite l'apprentissage par renforcement et le plongement des couches du réseau.

\section{Vue globale de la solution}



