\chapter*{Introduction générale}
\addcontentsline{toc}{chapter}{Introduction générale}
\markboth{Introduction générale}{Introduction générale}
\label{chap:introduction}
%\minitoc


L'apprentissage profond, ou deep learning, est une branche de l'intelligence artificielle (IA) 
qui a transformé de nombreux secteurs et industries, en particulier ces dernières années. 
Grâce à ses capacités avancées, l'apprentissage profond a permis des avancées significatives
dans des domaines tels que la reconnaissance d'image, la compréhension du langage naturel et 
la génération de données. En tant que composant essentiel de l'intelligence artificielle générative, 
l'apprentissage profond est utilisé pour créer divers types de données, 
y compris des textes, des images, des données tabulaires et des données séquentielles.

\medskip

Les architectures d'intelligence artificielle générative, telles que les GANs (Generative Adversarial Networks),
 les LLMs (Large Language Models) et les autoencodeurs variationnels (VAE), jouent un rôle crucial dans la génération 
 de données synthétiques. Ces modèles sont particulièrement efficaces pour augmenter les ensembles de données existants,
  ce qui est essentiel pour entraîner d'autres modèles de machine learning avec des ensembles de données plus diversifiés
   et représentatifs.


\medskip
Un domaine d'application particulièrement intéressant est celui des moteurs électriques, 
qui sont largement utilisés aujourd'hui et jouent un rôle crucial dans divers secteurs de l'industrie, 
notamment le transport. Ces moteurs, provenant de multiples fabricants et marques, nécessitent une
maintenance prédictive pour garantir leur bon fonctionnement et prolonger leur durée de vie. Cependant, 
la diversité des marques
et des modèles de moteurs pose un défi en termes de collecte de données suffisantes et
variées pour chaque type de moteur.


\medskip
Dans ce contexte, l'IA générative peut être utilisée pour générer des données
synthétiques qui couvrent un large éventail de moteurs électriques. En généralisant
sur toutes les variétés de moteurs existants, l'IA générative permet d'augmenter les
ensembles de données, ce qui est crucial pour entraîner des modèles de classification
et de prédiction plus précis et robustes. Ces modèles peuvent ensuite être utilisés pour
effectuer une maintenance prédictive efficace, réduisant ainsi les temps d'arrêt et les coûts de maintenance.


\medskip
Ce travail se concentrera sur l'application de l'IA générative pour la génération de 
données de type séries temporelles. Nous viserons à généraliser ces données pour qu'elles
représentent une large gamme de moteurs électriques. L'objectif final est de faciliter la 
maintenance prédictive de ces moteurs en utilisant des ensembles de données augmentés et diversifiés, 
permettant ainsi d'améliorer la fiabilité et l'efficacité des systèmes de maintenance.








 % orignsation de rapport 
\medskip

Dans cet article, nous présentons les méthodes d'intelligence artificielle générative
dans le contexte de l'augmentation de dataset pour la maintenance prédictive. 
Nous utilisons notamment les réseaux adversatifs génératifs (GAN) et les modèles de diffusion.
Les GAN, qui sont généralement composés d'un générateur et d'un discriminateur, 
permettent de générer des données synthétiques où le générateur crée des données
et le discriminateur évalue la qualité de ces données. Par ailleurs,
nous abordons les modèles de diffusion qui génèrent des données à partir d'un bruit gaussien. Nous détaillons les différentes étapes nécessaires pour générer des séries temporelles et discutons des méthodes d'évaluation pour apprécier la qualité des données générées par ces modèles. En outre, 
nous appliquons des techniques de traitement du signal pour visualiser les données dans le domaine fréquentiel.

\medskip
Nous commençons par une revue de la littérature sur l'apprentissage profond et 
les différentes architectures existantes. Nous y présentons également des modèles 
génératifs tels que les autoencodeurs variationnels (VAE), les GAN et les modèles de 
langage de grande taille (LLM). Le dernier chapitre de cette revue bibliographique est 
consacré aux bases de la maintenance prédictive, aux composants des moteurs
électriques, ainsi qu'aux techniques de traitement du signal comme la transformation de 
Fourier rapide (FFT).



