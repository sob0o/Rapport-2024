\chapter*{Conclusion et perspectives}
\addcontentsline{toc}{chapter}{Conclusion et perspectives}
\markboth{Conclusion et perspectives}{Conclusion et perspectives}
\label{chap:conclusion}
%\minitoc

Avec l'augmentation de la profondeur des architecture des réseaux de neurones, le nombre de calculs et la taille des réseaux augmentent également, ce qui rend leurs déploiement sur des appareils dotés d'un matériel limité très difficile et compliqué. L'élagage a émergé comme une approche pour réduire la complexité de ces réseaux profonds. Cependant, cette approche prend beaucoup de temps et nécessite des experts humains afin de bien élaguer un réseau. En raison de ses défauts, des méthodes automatiques utilisant l'apprentissage par renforcement sont apparues. Ces méthodes fournissent des résultats exceptionnels et ont la capacité à s'adapter à une grande variété d'environnements en utilisant des configurations appropriées. Cependant, les algorithmes d'apprentissage par renforcement ne peuvent pas prendre en entrée un réseau profond complet car il est très complexe. Il est donc nécessaire d'utiliser des structures moins complexes comme entrée telles que le plongement de graphe, de chouche, de noeuds, etc. Le plongement est un vecteur unidimentionnel qui garde seulement les informations importantes dans le réseau. En transformant ces données en un vecteur unidimensionnel, cet outil de plongement facilite grandement la capacité de l'agent d'apprentissage par renforcement à appréhender et à interagir avec son environnement.

Dans ce rapport, nous avons découvert les différentes techniques d'élagage des réseaux de neurones, ainsi que les types de plongements et les différents aspects des algorithmes d'apprentissage par renforcement. Nous avons commencé le rapport par une introduction au domaine d'apprentissage profond où nous avons vu quelques définitions et concepts de base, tels que les poids, les connexions, les types de réseaux de neurones, les différents types d'apprentissage (supervisé, non-supervisé, semi-supervisé et par renforcement), etc. Ensuite, nous avons présenté quelques concepts et algorithmes de l'apprentissage par renforcement, tels que les processus de décision de Markov, l'apprentissage Q profond, etc. Nous avons également parlée sur les graphes et les différentes techniques de plongement, ainsi que l'hypothèse du ticket de loterie et les types d'élagage des réseaux de neurones.

On a parlé de tout cela juste pour acquérir suffisamment de connaissances pour pouvoir élaborer une nouvelle approche d'élagage des réseaux de neurones profonds, en utilisant les techniques de plongement de couches et un algorithme d'apprentissage par renforcement complexe pour nous fournir les pourcentage d'élagage du modèle, pour arriver enfin à un modèle de taille considérablement réduite et avec une réduction minimale de la précision. Nous avons commencé par la présentation des modèles testés et le jeux de données utilisée pour les entraîner. Puis, nous avons présenté une vue global de la solution et ensuite les détails de chaque étape de la solution, qui commence par la construction du plongement et finit par le réglage fin. Enfin, nous avons vu les différents résultats d'application de notre méthode sur les modèles et nous les avons comparés avec les résultats de quelques méthodes performantes.

L'avantage le plus important de notre méthode est qu'elle permet de réduire considérablement la taille des modèles et le nombre de calculs, ce qui est idéal pour déployer ces modèles sur des appareils dotés d'un matériel limité. Toutefois, cet avantage est parfois accompagné d'une petite dégradation de la précision, ce qui nécessite de faire un réglage fin après l'élagage afin de restaurer partiellement la précision du modèle initial.

Même si notre méthode donne de très bons résultats, des améliorations sont encore possibles. Nous pouvons apporter ces améliorations à différentes parties de notre processus d’élagage. Par exemple, nous pouvons essayer de généraliser notre méthode à d'autres types d’architectures de réseau autres que les réseaux convolutifs et résiduels. Nous pouvons également modifier l'algorithme d'élagage de l'élagage des canaux vers un autre type d'élagage qui peut contribuer à augmenter la précision de nos modèles. Nous pouvons également essayer de faire d'autres types de plongement, tels que le plongement de l'ensemble du réseau, et utiliser d'autres algorithmes d'apprentissage par renforcement, tels que PPO ou A3C, ou même essayer un algorithme d'optimisation. Nous pouvons aussi essayer de faire l'élagage au début ou pendant l'entraînement afin d'éviter l'étape de réglage fin qui peut prendre du temps pour le ré-entraînement.
    
    

\section*{Perspectives}

    Lorem ipsum dolor sit amet, consectetur adipiscing elit. Proin posuere euismod neque, non semper nibh viverra sed. Praesent ut varius magna. Fusce ipsum ante, semper nec interdum at, semper et lacus. Nulla ultrices magna a fringilla finibus. Etiam sollicitudin blandit ante. Vivamus blandit rhoncus tincidunt. Morbi sit amet congue purus. Praesent interdum gravida congue. Donec fermentum dui fermentum maximus rutrum.:
    \medskip
    
    \renewcommand{\labelitemi}{$\bullet$}
    \begin{itemize}
        \item Le développement d'une application mobile pour l'animateur de zone :
        
        Lorem ipsum dolor sit amet, consectetur adipiscing elit. Proin posuere euismod neque, non semper nibh viverra sed. Praesent ut varius magna. Fusce ipsum ante, semper nec interdum at, semper et lacus. Nulla ultrices magna a fringilla finibus. Etiam sollicitudin blandit ante. Vivamus blandit rhoncus tincidunt. Morbi sit amet congue purus. Praesent interdum gravida congue. Donec fermentum dui fermentum maximus rutrum.
        
        \medskip
        
        \item L'amélioration de l'algorithme d'optimisation : 
        
        Lorem ipsum dolor sit amet, consectetur adipiscing elit. Proin posuere euismod neque, non semper nibh viverra sed. Praesent ut varius magna. Fusce ipsum ante, semper nec interdum at, semper et lacus. Nulla ultrices magna a fringilla finibus. Etiam sollicitudin blandit ante. Vivamus blandit rhoncus tincidunt. Morbi sit amet congue purus. Praesent interdum gravida congue. Donec fermentum dui fermentum maximus rutrum.
    \end{itemize}
    
    
    
\section*{Appréciation personnelle}

    Lorem ipsum dolor sit amet, consectetur adipiscing elit. Proin posuere euismod neque, non semper nibh viverra sed. Praesent ut varius magna. Fusce ipsum ante, semper nec interdum at, semper et lacus. Nulla ultrices magna a fringilla finibus. Etiam sollicitudin blandit ante. Vivamus blandit rhoncus tincidunt. Morbi sit amet congue purus. Praesent interdum gravida congue. Donec fermentum dui fermentum maximus rutrum.





    

%%% Local Variables: 
%%% mode: latex
%%% TeX-master: "isae-report-template"
%%% End: 

